\wt{1.6.1 Fundamental Estimate of Linear Algebra} No LI subset of $V$ has more elements than a generating set. Thus if $L \subset V$ a LI subset \& $E \subseteq V$ a generating set, then $|L| \le |E|$.
\wt{1.6.2 Steinitz Exchange Theorem} Let $L \subset V$ be a finite LI subset \& $E \subseteq V$ a generating set. Then there is an injection $\phi : L \hookrightarrow E$ s.t. $(E \setminus \phi(L)) \cup L$ also generates $V$.
\wl{1.6.3 Exchange Lemma} Let $M \subseteq V$ a LI subset, \& $E \subseteq V$ a generating subset, such that $M \subseteq E$. If $\vec{v} \in V \setminus M$ is a vector s.t. $M \cup \{\vec{v}\}$ is LI, then there exists $\vec{e} \in E \setminus M$ such that $\{E \setminus \{\vec{e}\}\} \cup \{\vec{v}\}$ spans $V$.
\wc{1.6.4 Cardinality of Bases} Let $V$ be finitely-generated.
(1) $V$ has a finite basis.
(2) $V$ cannot have an infinite basis.
(3) Any two bases of $V$ have the same number of elements.
\wde{1.6.5 Dimension} The cardinality of each basis of a finitely generated $V$ is called its dimension (denoted $\dim V$, or $\dim_{F}V$). If $V$ is not finitely generated, we say $\dim V = \infty$.
\wc{1.6.7 Cardinality Criterion for Bases} Let $V$ be finitely generated.
(1) Each LI subset $L \subset V$ has at most $\dim V$ elements, \& if $|L| = \dim V$ then $L$ is actually a basis.
(2) Each generating set $E \subseteq V$ has at least $\dim V$ elements, \& if $|E| = \dim V$ then $E$ is actually a basis.
\wc{1.6.8 Dimension Estimate for Vector Subspaces} A proper subspace of a finite VS has itself a strictly smaller dimension.
\wc{1.6.9 Remark on 1.6.8} If $U \subseteq V$ is a subspace of an arbitrary VS, then $\dim U \le \dim V$ \& if we have $\dim U = \dim V < \infty$ it follows that $U = V$.
\wde{1.6.9 Sum} If $U,W$ are subspaces of $V$, $U + W$ is the subspace $\langle U \cup W \rangle$ of $V$ generated by $U$ \& $W$ together.
\wt{1.6.10 The Dimension Theorem} $\dim(U + W) + \dim(U \cap W) = \dim U + \dim W$.
