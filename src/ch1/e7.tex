\subsection*{1.7 Linear Mappings}
\item Every composition of vector space homomorphisms is again a vector space homomorphism. In other words if $f : V \to W$ nd $g : U \to V$ are linear mappings, then so too is $f \circ g : U \to W$. Prove this.
\item Show that if $f : V \to W$ is a vector space isomorphism, then the inverse mapping $f^{-1} : W \to V$ is also a vector pace isomorphism. In particular, the automorphisms of a vector space $V$ form a subgroup of its permutation group. Thi group is caled the \emph{general linear group} or the \emph{automorphism} group of our vector pace $V$ and is denoted by $\text{GL}(V) = \Aut(V) \subseteq \Maps^{\times}(V, V)$. If I want to draw attention to the ground field, I will write $\Aut_{F}(V)$. These groups are amongs my all-time personal favourites.
\item Show that the image of a vector subspace under a linear mapping is again a vector subspace, and that the preimage of a vector subspace under a linear mapping is again a vector subspace.
\item Let $V$ be a vector space and $f \in \Endo(V)$ an endomorphism. Show that the fixed point set $V^{f} \subseteq V$ is a vector subspace.
\item (Homomorphisms from direct sums) Show that given vector spaces $V_{1}, \dots, V_{n}, W$ and linear mappings $f_{i} : V_{i} \to W$ then we can form a new linear mapping $f : V_{1} \oplus \dots \oplus V_{n} \to W$ by the recipe $f(v_{1}, \dots, v_{n}) = f_{1}(v_{1}) + \dots + f_{n}(v_{n})$. In this way, we even get a bijection $\Hom(V_{1}, W) \times \dots \times \Hom(V_{n}, W) \iso \Hom(V_{1}, \dots, V_{n} \to W)$, whose inverse can be written as $f \mapsto (f \circ in_{1}, \dots, f \circ in_{n})$.
\item (Homomorphisms for Products) Show that given vector spaces $V, W_{1}, \dots, W_{n}$ and linear mappings $g_{i} : V \to W_{i}$ then we can form a new linear mapping $g : V \to W_{1} \oplus \dots \oplus W_{n}$ by the recipe $g(v) = (g_{1}(v), \dots, g_{n}(v))$. In this way, we even get a bijection $\Hom(V, W_{1}) \times \dots \times \Hom(V, W_{n}) \iso \Hom(V, W_{1} \oplus \dots \oplus W_{n})$, whose inverse can be written as $g \mapsto (pr_{i} \circ g)_{i}$
\item How many vector subpaces are there in $\mathbb{R}^{2}$ that are sent to themselves under the reclection $(x,y) \mapsto (x, -y)$? Which vector subspaces in $\mathbb{R}^{3}$ are sent to themselves by the reflection $(x,y,z) \mapsto (x,y,-z)$? (Diagrams of reflections are in full notes)
\item Let $V,W$ be vector paces over a field $F$. Show that $\Hom_{F}(V, W)$ is a vector subspace of the set of all mappings $\Maps(V, W)$ from $V$ to $W$ with its vector space structure given similarly to def 1.5.15. Show that $\dim \Hom_{F}(V, W) = (\dim V) (\dim W)$ (using the convention $0 \cdot \infty = 0 = \infty \cdot 0$). In fact, it's possible to make a more sophisticated version of this equality using the relationship with cardinality given by lemma 1.7.8.
\item Let $V$ be a finite dimensional vector space, and let $U$ be a proper vector subspace. Show that there exists at east one (and in fact many different) vector subspaces of $V$ complementary to $U$. If you're brave, try to do this also for not necessarily finite dimensional vector spaces.
\item Show that each linear mapping from a vector subspace $U$ of a vector space $V$ to another vector space $W$, $f : U \to W$, can be extended to a linear mapping $\tilde{f} : U \to W$ on the whole vector space $V$.
