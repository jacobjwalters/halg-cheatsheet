\wl{3.2.1 Additive Inverses} Let $a, b \in R$. Then:
(1) $0a = 0 = a0$;
(2) $(-a)b = -(ab) = a(-b)$;
(3) $(-a)(-b) = ab$.
\wde{3.2.3 Multiple} Let $m \in \mathbb{Z}$. The $m$-th multiple $ma$ of $a \in$ an abelian group $R$ is $ma = a + \dots + a$ if $m > 0$, $0$ if $m=0$, and $(-m)a=-(ma)$.
\wl{3.2.4 Multiple Rules} Let $m, n \in \mathbb{Z}$. Then:
(1) $m(a + b) = ma + mb$;
(2) $(m + n)a = ma + na$;
(3) $m(na) = (mn)a$;
(4) $m(ab) = (ma)b = a(mb)$;
(5) $(ma)(nb) = (mn)(ab)$.
\wde{3.2.6 Unit} $a \in R$ is a unit if $\exists a^{-1} \in R$.
\wpr{3.2.9} The set $R^{\times}$ of units of $R$ forms a group under $\cdot$. e.g. if $R= \Mat(n; F)$ then $R^{\times} = \GL(n; F)$.
\wde{3.2.11 0-Divisor} $a \ne 0 \in R$ is a zero-divisor if $\exists b$ s.t. either $ab=0$ or $ba=0$.
\wde{3.2.12 Integral Domain} A non-zero comm. ring with no 0-divisors. These hold:
(1) $ab=0 \Rightarrow a = 0 \text{ or } b = 0$;
(2) $a \ne 0 \text{ and } b \ne 0 \Rightarrow ab \ne 0$. cf. lem1.2.4
\wpr{3.2.15 Cancellation} Let $R$ be an ID and $a, b, c \in R$. If $ab = ac$ and $a \ne 0$ then $b = c$.
\wpr{3.2.16} $\mathbb{Z}/m\mathbb{Z}$ is an ID iff. $m$ is prime.
\wt{3.2.17} Every \textbf{finite} integral domain is a field.
